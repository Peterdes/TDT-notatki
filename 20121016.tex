\documentclass[12pt]{article}
\usepackage{polski}
\usepackage[utf8]{inputenc}
\usepackage[T1]{fontenc}
\usepackage{amsmath}
\usepackage{amsfonts}
\usepackage{fancyhdr}
\usepackage{lastpage}
\usepackage{multirow}
\usepackage{amssymb}
\usepackage{amsthm}
\usepackage{textcomp}
\frenchspacing
\usepackage{fullpage}
\setlength{\headsep}{30pt}
\setlength{\headheight}{12pt}
%\setlength{\voffset}{-30pt}
%\setlength{\textheight}{730pt}
\pagestyle{myheadings}

\usepackage{tikz}
%\usepackage{tikz-cd}
\usetikzlibrary{arrows}

\newcommand{\bigslant}[2]{{\left.\raisebox{.2em}{$#1$}\middle/\raisebox{-.2em}{$#2$}\right.}}

\newcommand{\mf}[1]{{\mathfrak{#1}}}
\newcommand{\mb}[1]{{\mathbb{#1}}}
\newcommand{\mc}[1]{{\mathcal{#1}}}
\newcommand{\mr}[1]{{\mathrm{#1}}}
\newcommand{\Grass}{{\mathrm{Grass}}}
\newcommand{\Stief}{{\mathrm{Stief}}}

\newcounter{punkt}

\theoremstyle{plain}
\newtheorem{twierdzenie}[punkt]{Twierdzenie}
\newtheorem{twierdzeniebd}[punkt]{Twierdzenie (bez dowodu)}
\newtheorem{lemat}[punkt]{Lemat}

\theoremstyle{definition}
\newtheorem{definicja}[punkt]{Definicja}
\newtheorem{stwierdzenie}[punkt]{Stwierdzenie}
\newtheorem{stwierdzeniebd}[punkt]{Stwierdzenie (bez dowodu)}
\newtheorem{wniosek}[punkt]{Wniosek}

\theoremstyle{remark}
\newtheorem{uwaga}[punkt]{Uwaga}
\newtheorem{przyklad}[punkt]{Przykład}
\newtheorem{cytat}[punkt]{Cytat}


\markright{Piotr Suwara\hfill Topologia działania torusa: 16 października 2012 \hfill}
 
\begin{document}
 \begin{twierdzenie}
  $G$ zwarta, jeśli $U \subset X$ jest $G$-ekwiwariantnym otoczeniem $Gx$ oraz istnieje $G$-ekwiwariantna rektrakcja $p: U \to Gx$, to $U$ jest tubą wokół $Gx$ oraz $p^{-1}(x)$ jest slajsem.
 \end{twierdzenie}
 
 \begin{twierdzeniebd}[Chevalley]
  Każdą orbitę $G/H$ można zanurzyć ekwiwariantnie w pewną reprezentację liniową $G \to GL(V)$.
 \end{twierdzeniebd}
 
 \begin{twierdzeniebd}[Tietz-Gleason]
  $X$ normalna, $X \supset A$ domknięty $G$-niezmienniczy, $V$ reprezentacja $G$, $f:A \to V$ $G$-niezmiennicza. Wtedy istnieje $F$-niezmiennicze rozszerzenie $\tilde{f}:X \to V$.
 \end{twierdzeniebd}
 
 \begin{twierdzeniebd}[Mostow]
  $G$ zwarta grupa Lie, $X$ ma skończenie wiele typów orbitowych, $X$ metryczna skończonego wymiaru (tj. można zanurzyć w $\mb{R}^n$), to istnieje reprezentacja $V$ grupy $G$ taka, że $X$ zanurza się ekwiwariantnie w $V$.
 \end{twierdzeniebd}
 
 \begin{twierdzenie}
  $G$ zwarta grupa Lie, działa gładko na rozmaitości zwartej, to $G$ ma skończenie wiele typów orbitowych.
 \end{twierdzenie}
 
 \begin{twierdzeniebd}[Luny o slajsie]
  $X$ rozmaitość algebraiczna normalna (np. gładka), $G$ grupa reduktywna, $Gx$ domknięta. Wtedy istnieje $G_x$ przestrzeń $A$ oraz $G \times_{G_x} A \to X$ otoczenie w topologii etalnej, tzw. lokalnymi homeomorfizmami (topologia Grothendicka).
 \end{twierdzeniebd}
 
 \begin{twierdzeniebd}[Sumihiro o zanurzeniu]
  $X$ rozmaitość rzutowa normalna, $G$ grupa reduktywna, to istnieje reprezentacja $G$ i $G$-niezmiennicze zanurzenie $X \to \mb{P}(Y)$.
 \end{twierdzeniebd}
 
 {\bf Uniwersalne $G$-wiązki główne}
 
 \begin{twierdzenie}
  Niech $E \to B$ będzie wiązką główną taką, że $E$ jest przestrzenią ściągalną. Wtedy dla każdego CW-kompleksu $X$ i wiązki głównej $P\to X$ istnieje $f:X \to B$ takie, że $f^\ast E = P$. Poza tym $f$ jest jednoznaczne z dokładnością do homotopii. Przestrzeń $B$ oznaczamy $BG$, a $E$: $EG$.
 \end{twierdzenie}
 
 \begin{uwaga}
  $P$ normalna, $G$ zwarta Lie, działa wolno, to $P\to P/G$ jest wiązką główną.
 \end{uwaga}
 
 \begin{wniosek}
  Klasy izomorfizmów wiązek głównych dla $X$ CW-kompleksu odpowiadają elementom $[X, BG]$.
 \end{wniosek}
 
 \begin{wniosek}
  Jeśli $G$ ma model $BG$ będący CW-kompleksem, to $BG$ jest zdefiniowane z dokładnością do homotopii.
 \end{wniosek}
 
 \begin{uwaga}
  Jeśli $G$ Lie, to ma CW-model.
 \end{uwaga}












\end{document}
 
