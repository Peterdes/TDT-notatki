\documentclass[12pt]{article}
\usepackage{polski}
\usepackage[utf8]{inputenc}
\usepackage[T1]{fontenc}
\usepackage{amsmath}
\usepackage{amsfonts}
\usepackage{fancyhdr}
\usepackage{lastpage}
\usepackage{multirow}
\usepackage{amssymb}
\usepackage{amsthm}
\usepackage{stmaryrd}
\usepackage{textcomp}
\frenchspacing
\usepackage{fullpage}
\setlength{\headsep}{30pt}
\setlength{\headheight}{12pt}
%\setlength{\voffset}{-30pt}
%\setlength{\textheight}{730pt}
\pagestyle{myheadings}

\usepackage{tikz}
%\usepackage{tikz-cd}
\usetikzlibrary{arrows}

\newcommand{\bigslant}[2]{{\left.\raisebox{.2em}{$#1$}\middle/\raisebox{-.2em}{$#2$}\right.}}

\newcommand{\mf}[1]{{\mathfrak{#1}}}
\newcommand{\mb}[1]{{\mathbb{#1}}}
\newcommand{\mc}[1]{{\mathcal{#1}}}
\newcommand{\mr}[1]{{\mathrm{#1}}}
\newcommand{\Grass}{{\mathrm{Grass}}}
\newcommand{\Stief}{{\mathrm{Stief}}}

\newcounter{punkt}

\theoremstyle{plain}
\newtheorem{twierdzenie}[punkt]{Twierdzenie}
\newtheorem{twierdzeniebd}[punkt]{Twierdzenie (bez dowodu)}
\newtheorem{lemat}[punkt]{Lemat}

\theoremstyle{definition}
\newtheorem{definicja}[punkt]{Definicja}
\newtheorem{stwierdzenie}[punkt]{Stwierdzenie}
\newtheorem{stwierdzeniebd}[punkt]{Stwierdzenie (bez dowodu)}
\newtheorem{wniosek}[punkt]{Wniosek}

\theoremstyle{remark}
\newtheorem{uwaga}[punkt]{Uwaga}
\newtheorem{przyklad}[punkt]{Przykład}
\newtheorem{cytat}[punkt]{Cytat}


\markright{Piotr Suwara\hfill Topologia działania torusa: 30 października 2012 \hfill}
 
\begin{document}
 \begin{twierdzeniebd}[Iwasawa]
  $G$ spójna grupa Lie, to $G$ zawiera podgrupy $K$ maksymalną zwartą, $A$ abelową ($\simeq (\mb{R}_+^\ast)^n$), $N$ nilpotentną ($N_0=N, N_n=[N_{n-1},N_{n-1}], N_k=\{1\}$; $N \approx \mb{R}^n$ jako przestrzeń topologiczna) takie, że $K \times A \times N \to G, (k,a,n) \mapsto k \cdot a \cdot n$ jest homeomorfizmem.
 \end{twierdzeniebd}
 
 \begin{przyklad}
  $GL_n(\mb{R}) = O(n) \times \{\text{górnotrójkątne}\} = O(n) \times (\mb{R}_+^\ast)^n \times \{\text{ściśle górnotrójkątne}\}$
 \end{przyklad}
 
 \begin{uwaga}
  $G \simeq_{htp} K$, bo $G/K = AN \simeq_{top} \mb{R}^n$
 \end{uwaga}
 
 \begin{wniosek}
  $BG \simeq_{htp} BK$
 \end{wniosek}
 
 \begin{twierdzenie}
  $H^\ast(BGL_n(\mb{C})) \to H^\ast(BT)$ jest monomorfizmem, którego obraz jest równy $H^\ast(BT)^{\Sigma_n}$.
 \end{twierdzenie}
 
 \begin{wniosek}
  $H^\ast(BGL_n(\mb{C})) \simeq \mb{Z}[t_1, \ldots, t_n]^{\Sigma_n} = \mb{Z}[\sigma_1, \ldots, \sigma_n]$, w tym izomorfizmie $c_i \mapsto \sigma_i$, gdzie $c_i$ to \emph{$i$-ta klasa Cherna}.
 \end{wniosek}
 
 \begin{lemat}
  Niech $F_n=\{ 0 \subset V_1 \subset \ldots \subset V_n = V: \dim V_i=i\} \subset\prod_{i=1}^{n-1} \Grass_i(\mb{C}^n), L_i = V_i/V_{i-1}$, wtedy $c_1(L_i^\ast)$ generują $H^\ast(F_n)$.
 \end{lemat}
 
 \begin{definicja}
  $L=f^\ast(\gamma^\ast)$ gdzie $\gamma^\ast=\mc{O}(1)$ wiązka tautologiczna na $\mb{P}^\infty$.
  Wtedy niech $c_1(L) = f^\ast(x)$, gdzie $x \in H^2(\mb{P}^\infty)$ wyróżniony generator zadany przez hiperpowierzchnię.
 \end{definicja}
 
 \begin{twierdzenie}[Leray-Hirsch]
  Niech $F \hookrightarrow E \to B$ lokalnie trywialne rozwłóknienie, $H^\ast(F)$ wolne.
  Załóżmy, że istnieje transformacja $\phi: H^\ast(F) \to H^\ast(E)$ rozszczepiająca $i^\ast$, czyli $i^\ast \phi = \mr{id}_{H^\ast(F)}$.
  Wtedy kohomologie $H^\ast(E) \simeq H^\ast(B) \otimes H^\ast(F)$ (jako $H^\ast(B)$-moduły), $p^\ast b \cup \phi(f) \mapsfrom b \otimes f$.
 \end{twierdzenie}
 
 \begin{uwaga}
  Po drodze dostajemy, że $H^\ast(F_n) = H^\ast(BT) \otimes_{H^\ast(BGL_n(\mb{C}))} \mb{Z} = \mb{Z}[t_1, \ldots, t_n]/(\sigma_1, \ldots, \sigma_n)$.
 \end{uwaga}
 
 \begin{wniosek}
  $\mr{NatTrans}(G\text{-wiązki},H^\ast(\cdot, \mb{Z})) = H^\ast(BG)$
 \end{wniosek}
 
 \begin{wniosek}
  Dla $G=GL_n(\mb{C})$, $H^\ast(BG)=\mb{Z}[c_1, \ldots, c_n], c_i(E)=\sigma_i(t_1, \ldots, t_n)$, jeśli zaś $E=L_1 \oplus \ldots \oplus L_n$, to $t_i=c_1(L_i)$.
 \end{wniosek}
 
 \begin{wniosek}
  $c_n(E)=0$ jeśli $L_1$ jest trywialna.
 \end{wniosek}
 
 \begin{stwierdzenie}
  Jeśli wiązka $E$ ma przekrój, to $c_n(E)=0$.
 \end{stwierdzenie}
 
 \begin{stwierdzenie}
  $E \to X$ wiązka, Istnieje przestrzeń $Y$ i odwzorowanie $f:Y \to X$ takie, że $E$ rozszczepia się na wiązki liniowe nad $Y$, a odwzorowanie $f^\ast:H^\ast(X) \to H^\ast(Y)$ jest monomorfizmem.
 \end{stwierdzenie}















\end{document}
 
 
 
