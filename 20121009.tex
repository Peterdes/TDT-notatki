\documentclass[12pt]{article}
\usepackage{polski}
\usepackage[utf8]{inputenc}
\usepackage[T1]{fontenc}
\usepackage{amsmath}
\usepackage{amsfonts}
\usepackage{fancyhdr}
\usepackage{lastpage}
\usepackage{multirow}
\usepackage{amssymb}
\usepackage{amsthm}
\usepackage{stmaryrd}
\usepackage{textcomp}
\frenchspacing
\usepackage{fullpage}
\setlength{\headsep}{30pt}
\setlength{\headheight}{12pt}
%\setlength{\voffset}{-30pt}
%\setlength{\textheight}{730pt}
\pagestyle{myheadings}

\usepackage{tikz}
%\usepackage{tikz-cd}
\usetikzlibrary{arrows}

\newcommand{\bigslant}[2]{{\left.\raisebox{.2em}{$#1$}\middle/\raisebox{-.2em}{$#2$}\right.}}

\newcommand{\mf}[1]{{\mathfrak{#1}}}
\newcommand{\mb}[1]{{\mathbb{#1}}}
\newcommand{\mc}[1]{{\mathcal{#1}}}
\newcommand{\mr}[1]{{\mathrm{#1}}}
\newcommand{\Grass}{{\mathrm{Grass}}}
\newcommand{\Stief}{{\mathrm{Stief}}}

\newcounter{punkt}

\theoremstyle{plain}
\newtheorem{twierdzenie}[punkt]{Twierdzenie}
\newtheorem{twierdzeniebd}[punkt]{Twierdzenie (bez dowodu)}
\newtheorem{lemat}[punkt]{Lemat}

\theoremstyle{definition}
\newtheorem{definicja}[punkt]{Definicja}
\newtheorem{stwierdzenie}[punkt]{Stwierdzenie}
\newtheorem{stwierdzeniebd}[punkt]{Stwierdzenie (bez dowodu)}
\newtheorem{wniosek}[punkt]{Wniosek}

\theoremstyle{remark}
\newtheorem{uwaga}[punkt]{Uwaga}
\newtheorem{przyklad}[punkt]{Przykład}
\newtheorem{cytat}[punkt]{Cytat}


\markright{Piotr Suwara\hfill Topologia działania torusa: 9 października 2012 \hfill}
 
\begin{document}
 {\bf Konstrukcje ekwiwariantne}

 \begin{definicja}[skręcony produkt]
  Niech $G$ działa z prawej na $X$, z lewej na $Y$. Wtedy $X \times_G Y = X \times Y / \sim$, gdzie $(xg, y) \sim (x,gy)$, albo $=X \times Y / G$, gdzie $g(x,y) = (xg^{-1}, gy)$.
 \end{definicja}
 
 \begin{definicja}[przestrzeń z indukowanym działaniem]
  $H \subset G, X$ to $H$-przestrzeń, wtedy $G \times_H X$ to $G$-przestrzeń.
 \end{definicja}
 
 \begin{uwaga}
  $X$ lewa $G$-przestrzeń, prawa $H$-przestrzeń, $Y$ lewa $H$-przestrzeń, to $X \times_H Y$ ma strukturę $G$-przestrzeni.
 \end{uwaga}
 
 \begin{definicja}[produkt włóknisty]\raisebox{-0.5\height}{
  \begin{tikzpicture}[scale=1.4]
   \node (Prod) at (0,0.7) {$Y \times_X Z$};
   \node (Y) at (1,0.7) {$Y$};
   \node (Z) at (0,0) {$Z$};
   \node (X) at (1,0) {$X$};
   \path[->,>=angle 90]
   (Y) edge node[right]{$f$} (X)
   (Z) edge node[above]{$g$} (X)
   (Prod) edge node[above]{} (Y)
   (Prod) edge node[right]{} (Z);
  \end{tikzpicture}}, $Y \times_X Z = \{ (y,z) \in Y \times Z: f(y) = g(z) \}$.
 Jeśli $X,Y,Z$ to $G$-przestrzenie, a odwzorowania sa ekwiwariantne, to $Y \times_X Z$ ma naturalne działanie.
 \end{definicja}
 
 \begin{definicja}[przestrzeń indukowana]\raisebox{-0.5\height}{
  \begin{tikzpicture}[scale=1]
   \node (PB) at (0,1) {$f^\ast X = Y \times_{X/G} X$};
   \node (X) at (2.5,1) {$X$};
   \node (Y) at (0,0) {$Y$};
   \node (XG) at (2.5,0) {$X/G$};
   \path[->,>=angle 90]
   (X) edge node[right]{} (XG)
   (Y) edge node[above]{$f$} (XG)
   (PB) edge node[above]{} (X)
   (PB) edge node[right]{} (Y);
  \end{tikzpicture}}
 \end{definicja}
 
 \begin{definicja}[$G$-wiązki główne]
  $G$ działa na $E$ z prawej, działanie jest wolne, a $E \to E/G=B$ jest lokalnie trywialnym rozwłóknieniem, to $E\to B$ to $G$-wiązka główna.
 \end{definicja}
 
 \begin{twierdzeniebd}
  $E$ normalna, $G$ zwarta, to $E \to E/G$ jest lokalnie trywialne.
 \end{twierdzeniebd}
 
 \begin{definicja}[lokalnie trywialne rozwłóknienie]
  $E \to B$ lokalnie trywialne rozwłóknienie, jeśli istnieje pokrycie $U_i$ bazy $B$ takie, że $E|_{U_i} = E \times_B U_i \approx U_i \times G$.
 \end{definicja}
 
 \begin{uwaga}
  Zauważmy, że w takiej sytuacji jak wyżej,
  
  \begin{tikzpicture}[scale=1.5]
   \node (Ei) at (0,1) {$E|_{U_i}$};
   \node (Eij) at (3,1) {$E|_{U_i \cap U_j}$};
   \node (Ej) at (6,1) {$E|_{U_j}$};
   \node (Ui) at (0,0) {$U_i \times G$};
   \node (Uiji) at (2,0) {$(U_i \cap U_j) \times G$};
   \node (Uijj) at (4,0) {$(U_i \cap U_j) \times G$};
   \node (Uj) at (6,0) {$U_j \times G$};
   \node (gi) at (2,-0.5) {$(u,g)$};
   \node (gj) at (4,-0.5) {$(u,g_{ij}g)$};
   \path[->,=>angle 90]
   (Ei) edge (Ui)
   (Eij) edge (Uiji)
   (Eij) edge (Uijj)
   (Ej) edge (Uj)
   (Uiji) edge (Uijj);
   \path[|->,=>angle 90]
   (gi) edge (gj);
   \path[right hook->,=>angle 90]
   (Ei) edge (Eij)
   (Ui) edge (Uiji);
   \path[left hook->,=>angle 90]
   (Ej) edge (Eij)
   (Uj) edge (Uijj);
  \end{tikzpicture}

 \end{uwaga}
 
 \begin{definicja}[kocykl definiujący]
  \emph{Kocykl definiujący} to $(i,j) \mapsto g_{i,j}$ , gdzie $g_{ij}:U_i \cap U_j \to G$.
 \end{definicja}
 
 \begin{stwierdzenie}
  $\{g_{ij}\}$ spełniają warunek kocyklu $g_{ij} g_{jk} = g_{ik}$.
 \end{stwierdzenie}
 
 \begin{uwaga}
  Kocykl daje $G$-wiązkę.
 \end{uwaga}
 
 \begin{uwaga}
  $U_i \subset B$, $E|_{U_i}$ trywialna $\simeq U_i \times G$, to zamiana trywializacji $U_i \times G \to U_i \times G$ to po prostu $(u,g) \mapsto (u,h_i g)$, gdzie $h_i:U_i \to G$. Zmiana trywializacji na inną daje nowy kocykl $g_{ij}' = h_i^{-1} g_{ij} h_j$.
 \end{uwaga}
 
 \begin{stwierdzenie}
  Klasy izomorfizmu $G$-wiązek głównych odpowiadają granicy po pokryciach z kocykli podzielonych przez relację kobrzegowości, a to jest izomorficzne z $H^1(B, C(B,G))$, ale to wszystko tak na boku.
 \end{stwierdzenie}
 
 \begin{definicja}[przekształcenie wiązek głównych] $G$-niezmiennicze $f$, \raisebox{-0.4\height}{
  \begin{tikzpicture}[scale=1]
   \node (E) at (0,1) {$E$};
   \node (F) at (1.6,1) {$F$};
   \node (B) at (0.8,0) {$B$};
   \path[->,=>angle 90]
   (E) edge (B)
   (F) edge (B)
   (E) edge node[above]{$f$} (F);
  \end{tikzpicture}}.

 \end{definicja}

 
 \begin{stwierdzenie}
  Każde przekształcenie wiązek głównych jest izomorfizmem.
 \end{stwierdzenie}
 
 \begin{lemat}
  $E, F \to B \times I$ wiązki główne, $B$ parazwarte, jeśli $E|_{B \times \{0\}} \simeq F|_{B \times \{0\}}$, to $E \simeq F$.
 \end{lemat}
 
 W dowodzie powyższego założyliśmy, że $B$ to CW-kompleks.
 
 \begin{wniosek}
  \raisebox{-0.6\height}{\begin{tikzpicture}
   \node (PB) at (0,1) {$f^\ast E$};
   \node (E) at (2,1) {$E$};
   \node (Y) at (0,0) {$Y$};
   \node (X) at (2,0) {$X$};
   \path[->,>=angle 90]
   (E) edge node[right]{} (X)
   (Y) edge node[above]{$f$} (X)
   (PB) edge node[above]{} (E)
   (PB) edge node[right]{} (Y);
  \end{tikzpicture}}, $f^\ast E$ jest $G$-wiązką główną nad $Y$.
 \end{wniosek}
 
 \begin{twierdzenie}
  $f \simeq g:Y \to X \implies f^\ast E \simeq g^\ast E$.
 \end{twierdzenie}
 
 \begin{wniosek}
  Przyporządkowanie $X \mapsto$ zbiór klas izomorfizmu wiązek jest funktorem kontrawariantnym $hTop \to Set$.
 \end{wniosek}
 
 \begin{twierdzeniebd}
  Ten funktor jest ``prawie'' reprezentowalny, tzn. istnieje przestrzeń $BG$ (typu CW-kompleks, jeśli $G$ Lie) taka, że klasy homotopii $[X,BG] = $ klasy izomorfizmu $G$-wiązek głównych dla zwartego CW-kompleksu $X$.
 \end{twierdzeniebd}
 
 \begin{przyklad}
  $BS^1 = \mb{CP}^\infty$
  
  $BU(n) = \Grass_n(\mb{C}^\infty) = \bigcup_N \Grass_n(\mb{C}^N)$
 \end{przyklad}
 
 Lokalna struktura $G$-przestrzeni
 
 \begin{definicja}[tuba, slajs]
  $x \in X$, \emph{tubą} wokół orbity nazywamy stoczenie $U \supset Gx$ homeomorficzne z $G \times_{G_x} S$, gdzie $S \subset X, x \in S$ i $S$ jest $G_x$-niezmiennicze. $S$ nazywamy \emph{slajsem}.
 \end{definicja}
 
 \begin{twierdzenie}[Mostov, Wasserman]
  Jeśli $X$ normalna, $G$ zwarta Lie, to każda orbita ma tubę i slajs.
 \end{twierdzenie}
 
 \begin{lemat}
  Niech $V \to G/H$ wiązka wektorowa z działaniem $G$, która jest liniowa na włóknach (tj. \emph{$G$-wiązka wektorowa}), wtedy istnieje reprezentacja grupy $H$ na $W$ taka, że $V \approx G \times_H W \to G/H$.
 \end{lemat}
 
 \begin{twierdzeniebd}[o otoczeniu tubularnym]
  $Y \subset X$ podrozmaitość zwarta, to istnieje $\varepsilon>0$ taki, że $exp: NY \to X$ jest homeomorfizmem na wiązce dysków $D_\varepsilon \subset NY=TY^\perp \subset TX|_Y$.
 \end{twierdzeniebd}

























\end{document}
 
 
 
 
 
